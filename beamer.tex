%%%%%%%%%%%%%%%%%%%%%%%%%%%%%%%%%%%%%%%%%
% Beamer Presentation
% LaTeX Template
% Version 1.0 (10/11/12)
%
% This template has been downloaded from:
% http://www.LaTeXTemplates.com
%
% License:
% CC BY-NC-SA 3.0 (http://creativecommons.org/licenses/by-nc-sa/3.0/)
%
%%%%%%%%%%%%%%%%%%%%%%%%%%%%%%%%%%%%%%%%%

%----------------------------------------------------------------------------------------
%	PACKAGES AND THEMES
%----------------------------------------------------------------------------------------

\documentclass[t]{beamer}
\usepackage{etex}
\usepackage{pgfpages}
\mode<presentation> {

% The Beamer class comes with a number of default slide themes
% which change the colors and layouts of slides. Below this is a list
% of all the themes, uncomment each in turn to see what they look like.

%\usetheme{default}
%\usetheme{AnnArbor}
%\usetheme{Antibes}
%\usetheme{Bergen}
%\usetheme{Berkeley}
%\usetheme{Berlin}
%\usetheme{Boadilla}
%\usetheme{CambridgeUS}
%\usetheme{Copenhagen}
%\usetheme{Darmstadt}
%\usetheme{Dresden}
%\usetheme{Frankfurt}
%\usetheme{Goettingen}
%\usetheme{Hannover}
%\usetheme{Ilmenau}
%\usetheme{JuanLesPins}
%\usetheme{Luebeck}
\usetheme{Madrid}
%\usetheme{Malmoe}
%\usetheme{Marburg}
%\usetheme{Montpellier}
%\usetheme{PaloAlto}
%\usetheme{Pittsburgh}
%\usetheme{Rochester}
%\usetheme{Singapore}
%\usetheme{Szeged}
%\usetheme{Warsaw}

% As well as themes, the Beamer class has a number of color themes
% for any slide theme. Uncomment each of these in turn to see how it
% changes the colors of your current slide theme.

%\usecolortheme{albatross}
%\usecolortheme{beaver}
%\usecolortheme{beetle}
%\usecolortheme{crane}
%\usecolortheme{dolphin}
%\usecolortheme{dove}
%\usecolortheme{fly}
%\usecolortheme{lily}
%\usecolortheme{orchid}
%\usecolortheme{rose}
%\usecolortheme{seagull}
%\usecolortheme{seahorse}
\usecolortheme{whale}
%\usecolortheme{wolverine}

%\setbeamertemplate{footline} % To remove the footer line in all slides uncomment this line
%\setbeamertemplate{footline}[page number] % To replace the footer line in all slides with a simple slide count uncomment this line

%\setbeamertemplate{navigation symbols}{} % To remove the navigation symbols from the bottom of all slides uncomment this line
}

\usepackage{amsmath,amssymb,amsfonts,amscd}
\usepackage{stmaryrd}
\usepackage{latexsym}
\usepackage{comment}
\usepackage{subcaption}
\usepackage{multirow}
\usepackage{wasysym}
\usepackage{fancyhdr}
\usepackage{xspace}
\usepackage{graphicx}
\usepackage{mathtools}
\usepackage[all]{xy}
\usepackage{color}
\usepackage{algorithm}
\usepackage{algpseudocode}
\usepackage{booktabs}
\newcommand{\euler}{e}
\newcommand{\degree}{\ensuremath{^\circ}}
\newcommand{\Expect}{{\rm I\kern-.3em E}}
\newcommand{\HW}{\operatornamewithlimits{HW}}
\providecommand{\norm}[1]{\lVert#1\rVert}
\newcommand{\LFSR}{\operatornamewithlimits{LFSR}}
\DeclareMathOperator*{\argmin}{arg\,min}
\DeclareMathOperator*{\argmax}{arg\,max}
\DeclareMathOperator*{\trun}{trun}
\DeclareMathOperator*{\Var}{Var}
\DeclarePairedDelimiter\ceil{\lceil}{\rceil}
\DeclarePairedDelimiter\floor{\lfloor}{\rfloor}
\DeclarePairedDelimiter\abs{\lvert}{\rvert}

\providecommand{\innerprod}[1]{\langle#1\rangle}
%\setbeamertemplate{sections/subsections in toc}[sections numbered]
%\setbeamertemplate{subsections in toc}[sections numbered]

\setbeamertemplate{section in toc}
{\leavevmode\leftskip=2ex%
  \llap{%
    \usebeamerfont*{section number projected}%
    \usebeamercolor{section number projected}%
    \begin{pgfpicture}{-1ex}{0ex}{1ex}{2ex}
      \color{bg}
      \pgfpathcircle{\pgfpoint{0pt}{.75ex}}{1.7ex}
      \pgfusepath{fill}
      \pgftext[base]{\color{fg}\inserttocsectionnumber}
    \end{pgfpicture}\kern1.25ex%
  }%
  \inserttocsection\par}
\setbeamertemplate{subsection in toc}
  {\leavevmode\leftskip=2em$\bullet$\hskip1em\inserttocsubsection\par}



\newcommand{\N}{\mathbb N}
\newcommand{\Q}{\mathbb Q}
\newcommand{\F}{\mathbb F}
\newcommand{\C}{\mathbb C}
\newcommand{\Z}{\mathbb Z}
\newcommand{\R}{\mathbb R}
\DeclareMathOperator{\Li}{Li}
\newcommand{\tcr}{\textcolor{red}}
\newcommand{\tco}{\textcolor{orange}}
\newcommand{\tcy}{\textcolor{yellow}}
\newcommand{\tcg}{\textcolor{green}}
\newcommand{\tcb}{\textcolor{blue}}
\newcommand{\tcv}{\textcolor{violet}}
\newcommand{\tcp}{\textcolor{purple}}
\newcommand{\tcm}{\textcolor{magenta}}

\definecolor{burntorange}{RGB}{180,100,0}
\usecolortheme[RGB={180,100,0}]{structure} %Burnt Orange

\setbeamertemplate{items}[circle]
%----------------------------------------------------------------------------------------
%	TITLE PAGE
%----------------------------------------------------------------------------------------

\title[Lattice PUF]{Lattice PUF: A Strong Physical Unclonable Function Provably Secure against Machine Learning Attacks} % The short title appears at the bottom of every slide, the full title is only on the title page

\author[Ye Wang et al.]{Ye Wang, Xiaodan Xi, and Michael Orshansky} % Your name
\institute[UT ECE] % Your institution as it will appear on the bottom of every slide, may be shorthand to save space
{
University of Texas at Austin \\ % Your institution for the title page
\medskip
\textit{lhywang@utexas.edu} % Your email address
}
\date{\today} % Date, can be changed to a custom date

%\setbeamersize{text margin left=5pt,text margin right=30pt}
%\setbeamertemplate{itemize/enumerate body begin}{\large}
\begin{document}

\begin{frame}
\titlepage % Print the title page as the first slide
\end{frame}
\begin{frame}
\frametitle{Physical Unclonable Functions 101}
\begin{itemize}
\item	Silicon fingerprint via challenge-response pair (CRP)
	\begin{itemize}
	\item	Entropy from process variation
	\item	Invisible when power off and secure against physical attacks
	\item	Secure key generation and device authentication
	\end{itemize}
\item	Strong PUFs provide a much larger CRP space than Weak PUFs 
\begin{figure}
\centering
\includegraphics[width=.8\linewidth]{fig/StrongPUF.pdf}
\end{figure}
    \begin{itemize}
        \item Weak PUFs: SRAM PUF, Butterfly PUF
        \item Strong PUFs: Arbiter PUF, Ring-Oscillator (RO) PUF, controlled PUF
    \end{itemize}
\end{itemize}
\end{frame}

\begin{frame}
\frametitle{Machine Learning Attacks on Strong PUFs}
\begin{itemize}
    \item CRPs can be modelled by function $r=f(\mathbf{c})$
        \begin{itemize}
            \item Linear model: Arbiter PUF, RO PUF
            \item Non-linear model: XOR Arbiter PUF, SCA PUF
        \end{itemize}
    \item Flow of ML attacks:
    \begin{figure}
    \centering
    \includegraphics[width=.8\linewidth]{fig/ML Attack.pdf}
    \end{figure}
        \begin{itemize}
            \item Collect a bunch of CRPs for given PUF 
            \item Train a mathematical model $r = f'(\mathbf{c})$
            \item Use $f'$ to fake $f$
        \end{itemize}
\end{itemize}
\end{frame}

\begin{frame}{Problem: Can We Engineer a ML Resistant PUF?}
\begin{itemize}
    \item Arbiter PUF successfully attacked by support vector machine (SVM)
    \item Arbiter PUF variants by strengthening non-linearity successfully attacked by advanced ML attacks:
        \begin{itemize}
            \item Bi-stable Ring PUF, Feed-forward PUF, Lightweight secure PUF attacked by
            \item Interpose-PUF, XOR Arbiter PUF attacked by deep learning
        \end{itemize}
    \item Empirically-demonstrated ML resistance for PUFs with intrinsic nonlinearity 
    \item ML resistance via established cryptographic primitives
        \begin{itemize}
            \item Controlled PUF, AES PUF, computational fuzzy extractor
        \end{itemize}
\end{itemize}    
\end{frame}

\begin{frame}{Our Contribution}
\begin{itemize}
    \item A strong PUF with provable ML resilience derived from lattice cryptography.
    \item Efficient hardware implementation via distributional relaxation
\end{itemize}    
\end{frame}

\begin{frame}{Formal Definition of ML Resistance}
\begin{itemize}
    \item Intuitively, good measure of ML hardness need:
        \begin{itemize}
            \item Easy to learn: a learning algorithm can find a model with good quality, using a small number of CRPs, in a short time
            \item Hard to learn: \textbf{all possible learning algorithms fail, or they require a large number of CRPs, or a long running time}
        \end{itemize}
    \item Probably approximately correct (PAC) model is the only known formal framework for provable notion of ML resistance
    \begin{itemize}
%            \item Definition: strong PUF $\mathcal{F}$ is PAC-learnable if there exists a polynomial-time algorithm $\mathcal{A}$ such that $\forall \epsilon > 0$, $\forall \delta >0$, for any fixed CRP distribution $\mathcal{D}$, and $\forall f\in\mathcal{F}$, given a training set of size $m$, $\mathcal{A}$ produces a candidate model $h\in\mathcal{H}$ with probability of, at least, $1-\delta$ such that
%\begin{equation*}
%\Pr_{(\mathbf{c},r)\sim\mathcal{D}}[f(\mathbf{c})\neq h(\mathbf{c})] < \epsilon.
%\end{equation*}
        \item Definition: strong PUF $\mathcal{F}$ is ML-resistant if it is not PAC-learnable
        \item k-bit concrete hardness: if the current best ML attack requires $2^k$ CPU operations 
    \end{itemize}
    \item Challenge: are there known PAC non-learnable functions?
\end{itemize} 
\end{frame}

\begin{frame}{Decryption Functions Are not PAC Learnable}
\begin{itemize}
    \item A typical public-key cryptosystem
    \begin{figure}
    \centering
    \includegraphics[width=.8\linewidth]{fig/PublicKeyCrypto.pdf}
    \end{figure}
\end{itemize}    
\end{frame}

\begin{frame}{Decryption Functions Are not PAC Learnable}
\begin{itemize}
    \item Chosen plaintext attacks    
    \begin{figure}
    \centering
    \includegraphics[width=.8\linewidth]{fig/CPA.pdf}
    \end{figure}
    \item Security against chosen plaintext attacks
\end{itemize}    
\end{frame}

\begin{frame}{Decryption Functions Are not PAC Learnable}
\begin{itemize}
    %\item CRP space: challenge $\leftrightarrow$ ciphertext, response $\leftrightarrow$ plaintext (0/1)
    \item A successful learning algorithm could help construct a successful chosen plaintext attack
    \begin{figure}
    \centering
    \includegraphics[width=.8\linewidth]{fig/MLasDistinguisher.pdf}
    \end{figure}
\end{itemize}    
\end{frame}

\begin{frame}{Decryption Functions Are not PAC Learnable}
\begin{itemize}
    \item Theorem: If a public-key cryptosystem is secure against chosen plaintext attacks, then its decryption functions are not PAC-learnable (under the ciphertext input distribution).
    \item A PUF implemented using decryption function will be provably ML resistant 
    \item Which cryptosystem and decryption function to pick?
\end{itemize}
\end{frame}

\begin{frame}{Lattice Problems}
\begin{itemize}
%    \item A lattice $\mathcal{L}(\mathbf{V})$ is a set of integral linear combinations of a given basis $\mathbf{V}=\{\mathbf{v}_1,\mathbf{v}_2,\ldots, \mathbf{v}_n\}$:
    \item A lattice is a set of integral linear combinations of a given basis:
%\begin{equation*}
%\mathcal{L}(\mathbf{V}) = \{a_1\mathbf{v}_1 + a_2\mathbf{v}_2+\ldots a_n\mathbf{v}_n: \: \forall a_i \in \mathbb{Z}\}.
%\end{equation*}
    \begin{figure}
    \centering
    \includegraphics[width=.5\linewidth]{fig/Lattice.pdf}
    \end{figure}
    \item Several lattice based problems are assumed to be intractable 
    \begin{itemize}
        \item (Approximate) shortest vector problem (SVP), Approximate SIVP
    \end{itemize}
\end{itemize}
\end{frame}

\begin{frame}{Learning-With-Errors (LWE) Problem}
\begin{itemize}
    \item LWE problem: distinguish $(\mathbf{a}_i,b_i)$'s from a uniform distribution
    \begin{align*}
        b_1 &= \innerprod{\mathbf{a}_1,\mathbf{s}} + e_1\\
        b_2 &= \innerprod{\mathbf{a}_2,\mathbf{s}} + e_2\\
        \ldots\\
        b_m &= \innerprod{\mathbf{a}_m,\mathbf{s}} + e_m
    \end{align*}
    \begin{itemize}
        \item Fixed secret $\mathbf{s}\in \mathbb{Z}_q$, $\mathbf{a}_i\in \mathbb{Z}^n_q$ sampled uniformly
        \item $e_i \in Z_q$ from a discrete Gaussian distribution $\bar{\Psi}_\alpha$
    \end{itemize}
    \item LWE problem is intractable based on hardness assumption of several lattice problems
        \begin{itemize}
            \item LWE with $q=256$, $n = 137$, $\alpha = 2.20\%$ requires $2^{128}$ CPU operations to solve 
        \end{itemize}
\end{itemize}
\end{frame}

\begin{frame}{Learning-With-Errors (LWE) Cryptosystem}
\begin{itemize}
    \item Private key: 
    \item Public key:
    \item Encryption function:
    
    \item Decryption function:
\end{itemize}
\end{frame}

\begin{frame}{Overview of Lattice PUF Design}
\begin{figure}
    \centering
    \includegraphics[width = 0.95\linewidth]{./fig/TopLevelDesign.pdf} 
\end{figure}    
\begin{itemize}
    \item Core module: LWE decryption function
    \item Physically obfuscated key (POK) to implement secret $\mathbf{s}$
    \item LFSR-based PRNG to reduce challenge size
    \item Self-incrementing counter to prevent challenge manipulation attack
\end{itemize}
\end{frame}

\begin{frame}{Core Module: LWE Decryption Function}
 \begin{figure}
    \centering
    \includegraphics[width = 0.7\linewidth]{./fig/LWEDec.pdf} 
\end{figure}   
\begin{itemize}
    \item Challenge $\mathbf{c}$ mapping to $(\mathbf{a},b)$ via binary vector flattening
\begin{align*}
a_i &= \sum_{j=0}^{\log q-1}c_{(i-1)\log q+j}2^j,\; \forall i\in \{1,2,\ldots,n\}, \\
b &= \sum_{j=0}^{\log q-1}c_{n\log q+j}2^j. 
\end{align*}    
%    \item Modulo operation naturally implemented by fixed-length MAC 
%   \item Quantizer implemented by a integer comparator
\end{itemize}
\end{frame}

\begin{frame}{Parameter Selection in LWE Decryption Function}
\begin{itemize}
\item Major implementation costs: number of challenge and secret bits needed 
    \begin{itemize}
        \item Both scales with $n$ and $\log q$
    \end{itemize}
\item Output errors include environmental errors of secret bits and decryption errors
     \begin{itemize}
        \item POK failure rate needs to be low ($10^{-6}$) since single bit-flip completely changes the CRP behavior of LWEDec
        \item Decryption error due to error terms scales with $m$ and $\alpha$
    \end{itemize}   
\item Concrete ML hardness established by state-of-theart attacks on the LWE cryptosystem scales with $n$ and $\alpha$
%    \begin{itemize}
%        \item Set it to 128bit
%    \end{itemize}
\end{itemize}
\end{frame}

\begin{frame}{Cost vs. Reliability with 128-Bit Concrete Hardness}
\begin{figure}
\centering
\includegraphics[width=.7\linewidth]{fig/DecErrVsSecretBits.pdf}
\end{figure}   
\begin{itemize}
    \item $n = 160$, $q = 256$, $m = 256$ and $\alpha = 2.20\%$ $\rightarrow$ $128$-bit concrete hardness and decryption error rate of $1.26\%$.
\end{itemize}
\end{frame}

\begin{frame}{Challenge Compression via Distributional Relaxation}
\begin{itemize}
    \item  High ratio (1288:1) of challenge length to response length limits its practical use 
%    \begin{itemize}
%        \item In direct authentication, 100 bits of response requires 128.8K challenge bits transmitted
%    \end{itemize}
    \item Replacing $\mathbf{a}=\mathbf{A}^T\mathbf{x}$ by uniformly sampled $\mathbf{a}^*$ preserves the security properties []
 \begin{equation*}
    \begin{cases}
    \mathbf{a}= \mathbf{A}^T\mathbf{x}\\
    b = (\mathbf{A}^T\mathbf{x})^T\mathbf{s}+\mathbf{e}^T\mathbf{x}+r\floor{q/2}
    \end{cases}\rightarrow\;
    \begin{cases}
    \mathbf{a}^*\\
    b^*=\mathbf{a}^{*T}\mathbf{s}+\mathbf{e}^T\mathbf{x}+r\floor{q/2}
    \end{cases}
\end{equation*}   
\item Space-efficient LWE allows using PRNG to generate $\mathbf{a}^\prime$ with a seed $\text{seed}_{\mathbf{a}^\prime}$ and the corresponding $b^\prime$ becomes
\begin{align*}
    b^\prime&=(\mathbf{a}^\prime)^T\mathbf{s}+\mathbf{e}^T\mathbf{x}+r\floor{q/2}\\
    &= \LFSR{(\text{seed}_{\mathbf{a}^\prime})}^T\mathbf{s}+\mathbf{e}^T\mathbf{x}+r\floor{q/2}.
\end{align*}
    \begin{itemize}
        \item Only need to transmit short seeds
        \item Concrete ML hardness is still preserved
    \end{itemize}
\end{itemize}
\end{frame}

\begin{frame}{Counter Measure for Active Attack}
\begin{itemize}
    \item ML attack belongs to passive attacks
    \item LWE decryption function vulnerable to an active input challenges manipulation attack
    TODO: a figure shows how this attack works
    \item Embed a self-incrementing counter into the challenge seed []
    \begin{itemize}
        \item  restricts the attacker’s ability to completely control input challenges
    \end{itemize}
\end{itemize}
\end{frame}

\begin{frame}{Statistical Properties of Lattice PUF}
   \begin{figure}
  \begin{subfigure}[b]{0.45\textwidth}
    \includegraphics[width=\textwidth]{fig/Uniformity.pdf}
    \caption{Uniformity}
    \label{fig:1}
  \end{subfigure}
  %
  \begin{subfigure}[b]{0.5\textwidth}
    \includegraphics[width=\textwidth]{fig/UniquenessReliability.pdf}
    \caption{Uniqueness and Reliability}
    \label{fig:2}
  \end{subfigure}
\end{figure} 
\begin{itemize}
    \item Evaluated using 1000 randomly instantiated lattice PUFs, each with 1000 randomly generated challenges
\end{itemize}
\end{frame}

\begin{frame}{Empirical ML Resistance}
    \begin{figure}
  \begin{subfigure}[b]{0.48\textwidth}
    \includegraphics[width=\textwidth]{fig/DNNAllPUFs.pdf}
    %\caption{Uniformity}
    \label{fig:1}
  \end{subfigure}
  %
  \begin{subfigure}[b]{0.48\textwidth}
    \includegraphics[width=\textwidth]{fig/MLAttackLatticePUF.pdf}
    %\caption{Uniqueness and Reliability}
    \label{fig:2}
  \end{subfigure}
  
\end{figure}   
\begin{table}[t!]
\centering
	%\caption{Various configuration for DNN attacks.}
	\label{table:DNNSetting}
	\resizebox{.6\textwidth}{!}{
\begin{tabular}{|l|l|l|l|l|l|}
\hline
Setup & \begin{tabular}[c]{@{}l@{}}Hidden\\ Layers\end{tabular} & \begin{tabular}[c]{@{}l@{}}Neurons\\ per Layer\end{tabular} & \begin{tabular}[c]{@{}l@{}}Challenge \\ Distribution\end{tabular} & \begin{tabular}[c]{@{}l@{}}Input \\ Format\end{tabular} & \begin{tabular}[c]{@{}l@{}}Prediction\\ Error\end{tabular} \\ \hline
DNN-1     & 4                                                       & 100                                                         & PRNG                                                              & Binary                                                  & 49.86\%                                                       \\ \hline
DNN-2     & 4                                                       & 100                                                         & PRNG                                                              & Real                                                    & 49.84\%                                                       \\ \hline
DNN-3     & 4                                                       & 100                                                         & Ciphertext                                                        & Binary                                                  & 49.76\%                                                       \\ \hline
DNN-4     & 6                                                       & 100                                                         & PRNG                                                              & Binary                                                  & 49.80\%                                                       \\ \hline
DNN-5     & 4                                                       & 200                                                         & PRNG                                                              & Binary                                                  & 49.87\%                                                       \\ \hline
\end{tabular}
}
\end{table}
\end{frame}

\begin{frame}{Hardware Implementation Results}
\begin{itemize}
    \item Entire design, PUF logic and fuzzy extractor (FE), Synthesized, configured, and tested on a Xilinx Spartan-6 FPGA 
    \begin{itemize}
        \item Raw SRAM bits not included
    \end{itemize}
    \item  FE design based on error-correcting codes (ECC): 
\begin{figure}
\centering
\includegraphics[width=.5\linewidth]{fig/POK.pdf}
\end{figure}    
    \begin{itemize}
        \item Key reconstruction of 1280 bits with
targeted failure rate $10^{-6}$
        \item Concatenation code: repetition code + shortened BCH code
        \item Explore design costs with BER $1\%$, $5\%$, $10\%$, and $15\%$
    \end{itemize}
\end{itemize}
    
\end{frame}

\begin{frame}{Hardware Implementation Results: PUF Logic}
\begin{itemize}
    \item Reference implementation on Spartan-6 FPGA
    \begin{table}[t!]
    %\caption{Reference lattice PUF implementation on Xilinx Spartan-6 FPGA.}
    \label{table:fpga_result}
    \centering
    \def\arraystretch{1.1}
    \subfloat[Area consumption]{
        \resizebox{0.3\linewidth}{!}{
            \begin{tabular}{|c|c|}
            \hline
            \textbf{Module}         & \textbf{Size [slices]} \\ \hline
            LFSR                    & 27            \\ \hline
            LWEDec                  & 2             \\ \hline
            Controller              & 16            \\ \hline
            \textit{Total}          & 45            \\ \hline
            \end{tabular}
            \label{table:fpga_utilization}
        }
    }
    \hspace{1em}
    \subfloat[Runtime]{
        \resizebox{0.5\linewidth}{!}{
            \begin{tabular}{|c|c|}
            \hline
            \textbf{Step}                               & \textbf{Time [$\mu$s]} \\ \hline
            Seed $\text{seed}_{\mathbf{a}'}||t$ load for LFSR    & 8             \\ \hline
            1-bit decryption from LWEDec                  & 44            \\ \hline
            \textit{Total} @ 33 \textit{MHz}            & 52            \\ \hline
            \end{tabular}
            \label{table:fpga_timing}
        }
    }
\end{table}
\item Lightweight compared with other strong PUFs
\begin{table}[t!]
\centering
	%\caption{Hardware implementation costs of strong PUFs.}
	\label{table:hardware_puf}
	\def\arraystretch{1.1}
	\resizebox{0.6\linewidth}{!}{
        \begin{tabular}{|c|c|c|}
        \hline
        \textbf{Design}                                         & \textbf{Platform}  & \textbf{PUF Logic [Slices]} \\ \hline
        POK+AES []                    & Spartan 6 & 80                 \\ \hline
        Controlled PUF []             & Spartan 6 & 127                \\ \hline
        CFE-based PUF []  & Zynq-7000 & 9,825              \\ \hline
        Lattice PUF                                             & Spartan 6 & 45                 \\ \hline
        \end{tabular}
    }
    \vspace{-1em}
\end{table}

\end{itemize}
\end{frame}

\begin{frame}{Hardware Implementation Results: FE}
\begin{itemize}
    \item Configuration of error-correcting codes
    \begin{table}
    \centering
	%\caption{Configuration of error-correcting codes.}
	\vspace{0.5em}
	\label{table:ecc}
	\def\arraystretch{1.1}
	\resizebox{0.65\linewidth}{!}{
        \begin{tabular}{|c|c|c|c|c|c|}
        \hline
        \multirow{2}{*}{\begin{tabular}[c]{@{}c@{}}\textbf{Raw BER}\\  \textbf{(\%)}\end{tabular}} & \multicolumn{2}{c|}{\textbf{Error-Correcting Code}}                                 & \multirow{2}{*}{\textbf{Raw POKs}}  \\ \cline{2-3} 
                                                                                 & Outer code   & Inner code & \\ \hline
        1                                                                        & {[}236, 128, 14{]}  & N/A            & 2,360  \\ \hline
        5                                                                        & {[}218, 128, 11{]} & {[}3, 1, 1{]}  & 6,540  \\ \hline
        10                                                                       & {[}220, 128, 12{]} & {[}5, 1, 2{]}  & 11,000  \\ \hline
        15                                                                       & {[}244, 128, 15{]} & {[}7, 1, 3{]}  & 17,080   \\ \hline
        \end{tabular}
    }
\end{table}

    \item Hardware utilization of FE on Spartan 6 FPGA.
    \begin{table}
    \centering
    %\caption{Hardware utilization in FE design on Spartan 6 FPGA.}
    \label{table:hardware_fe}
	\resizebox{0.8\linewidth}{!}{
        \begin{tabular}{|c|c|c|c|c|c|c|c|c|c|}
        \hline
        \multirow{2}{*}{\begin{tabular}[c]{@{}c@{}}\textbf{Raw BER}\\  \textbf{(\%)}\end{tabular}} & \multicolumn{3}{c|}{\textbf{Outer Code}} & \multicolumn{3}{c|}{\textbf{Inner Code}} & \multicolumn{3}{c|}{\textbf{Total}} \\ \cline{2-10} 
                               & Reg      & LUT      & Slice     & Reg      & LUT      & Slice     & Reg    & LUT    & Slice    \\ \hline
        1                      & 905      & 893      & 276       & 0        & 0        & 0         & 905    & 893    & 276      \\ \hline
        5                      & 730      & 688      & 232       & 0        & 1        & 1         & 730    & 689    & 233      \\ \hline
        10                     & 785      & 740      & 243       & 0        & 3        & 2         & 785    & 743    & 245      \\ \hline
        15                     & 973      & 913      & 326       & 0        & 7        & 3         & 973    & 920    & 329      \\ \hline
        \end{tabular}
    }
\end{table}
\end{itemize}
\end{frame}

\begin{frame}{Conclusions}
\begin{itemize}
    \item  A strong PUF provably secure against machine learning (ML) attacks
    \begin{itemize}
        \item Security derived from hardness assumptions on lattice problems
        \item Exhibit empirical ML resistance even with advanced DNN attacks
    \end{itemize}
    \item Excellent uniformity, uniqueness, and reliability.
    \item Lightweight hardware implementation via distributional relaxation
\end{itemize}    
\end{frame}

\begin{frame}
\frametitle{Publication}
\begin{itemize}
\footnotesize{
\item	\textbf{Y. Wang}, X. Xi, A. Aysu, and M. Orshansky. ``Lattice PUF: A Provably Post-Quantum Machine Learning Resistant Strong Physical Unclonable Function,'' Under review of \textit{Conference on Cryptographic Hardware and Embedded Systems (CHES)}, 2018
\item	\textbf{Y. Wang} and M. Orshansky. ``Efficient Helper Data Reduction in SRAM PUFs via Lossy Compression,'' In \textit{Design, Automation and Test in Europe (DATE)}, 2018
\item	A. Aysu, \textbf{Y. Wang}, P. Schaumont, and M. Orshansky. ``A New Maskless Debiasing Method for Lightweight Physical Unclonable Functions,'' In \textit{International Symposium on Hardware Oriented Security and Trust (HOST)}, 2017
\item	\textbf{Y. Wang}, C. Caramanis, and M. Orshansky. ``Exploiting Randomness in Sketching for Efficient Hardware Implementation of Machine Learning Applications,'' In \textit{International Conference on Computer-Aided Design (ICCAD)}, 2016
\item	M. Li, \textbf{Y. Wang}, and M. Orshansky. ``A Monte Carlo Simulation Flow for SEU Analysis of Sequential Circuits,'' In \textit{Design Automation Conference (DAC)}, 2016
}
\end{itemize}
\end{frame}

\begin{frame}
\frametitle{Publication}
\begin{itemize}
\footnotesize{
\item	\textbf{Y. Wang}, C. Caramanis, and M. Orshansky. ``PolyGP:Improving GP-Based Analog Optimization through Accurate High-Order Monomials and Semidefinite Relaxation,'' In \textit{Design, Automation and Test in Europe (DATE)}, 2016
\item	\textbf{Y. Wang}, C. Caramanis, and M. Orshansky. ``PolyGP:Improving GP-Based Analog Optimization through Accurate High-Order Monomials and Semidefinite Relaxation,'' In \textit{International Workshop on Frontiers in Analog CAD (FAC)}, 2015
\item	\textbf{Y. Wang}, M. Li, X. Yi, Z. Song, M. Orshansky, and C. Caramanis. ``A Novel Power Grid Reduction Method Based on L1 Regularization,'' In \textit{Design Automation Conference (DAC)}, 2015
\item	\textbf{Y. Wang}, C. Caramanis, and M. Orshansky. ``Enabling Efficient Analog Synthesis by Coupling Sparse Regression and Polynomial Optimization,'' In \textit{Design Automation Conference (DAC)}, 2014
\item	\textbf{Y. Wang} and M. Orshansky. ``Reducing Amount of Helper Data in Silicon Physical Unclonable Functions via Lossy Compression without Production-Time Error Characterization,'' \textit{US patent pending}
}
\end{itemize}
\end{frame}

%\input{education}


\begin{frame}[c]
\Huge{\centerline{The End}}
\Huge{\centerline{Questions?}}
\end{frame}


\appendix
%\input{Appendix}


%------------------------------------------------
%\begin{frame}[allowframebreaks]{References}
%\frametitle{References}
%\footnotesize{
%\bibliographystyle{amsalpha}
%\bibliography{refs}
%}
%\end{frame}

%------------------------------------------------
%----------------------------------------------------------------------------------------

\end{document} 